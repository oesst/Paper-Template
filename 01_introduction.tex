\section*{Introduction}


What is the question you are going to answer?
\begin{itemize}
    \item Why is it an exciting question (basically the problem statement)
    \item Why is this question important to our field
    \item How is my work going to help to answer it?
\end{itemize}

"I have a major question. How did I get to this question? Why is that important? That’s essentially the introduction. And then I have the second part which is this idea that now I’m going to lead you through how I answered it, and that’s our methods and results. So, I think the story part comes in putting your work into context of the field, of other people’s work, of why it’s important, and it’ll make your results much more compelling" \cite{mensh2017ten}. 


Readers might look at the title, they might skim your abstract and they might look at your figures, so we try to make our figures tell the story as much as possible.


\paragraph{How to structure a paragraph}
"For the whole paper, the introduction sets the context, the results present the content and the discussion brings home the conclusion" \cite{mensh2017ten}..

"In each paragraph, the first sentence defines the context, the body contains the new idea and the final sentence offers a conclusion" \cite{mensh2017ten}..




\paragraph{From 'Ten Simple Rules for structuring papers'}
"The introduction highlights the gap that exists in current knowledge or methods and why it is important. This is usually done by a set of progressively more specific paragraphs that culminate in a clear exposition of what is lacking in the literature, followed by a paragraph summarizing what the paper does to fill that gap.

As an example of the progression of gaps, a first paragraph may explain why understanding cell differentiation is an important topic and that the field has not yet solved what triggers it (a field gap). A second paragraph may explain what is unknown about the differentiation of a specific cell type, such as astrocytes (a subfield gap). A third may provide clues that a particular gene might drive astrocytic differentiation and then state that this hypothesis is untested (the gap within the subfield that you will fill). The gap statement sets the reader’s expectation for what the paper will deliver.

The structure of each introduction paragraph (except the last) serves the goal of developing the gap. Each paragraph first orients the reader to the topic (a context sentence or two) and then explains the “knowns” in the relevant literature (content) before landing on the critical “unknown” (conclusion) that makes the paper matter at the relevant scale. Along the path, there are often clues given about the mystery behind the gaps; these clues lead to the untested hypothesis or undeveloped method of the paper and give the reader hope that the mystery is solvable. The introduction should not contain a broad literature review beyond the motivation of the paper. This gap-focused structure makes it easy for experienced readers to evaluate the potential importance of a paper—they only need to assess the importance of the claimed gap.

The last paragraph of the introduction is special: it compactly summarizes the results, which fill the gap you just established. It differs from the abstract in the following ways: it does not need to present the context (which has just been given), it is somewhat more specific about the results, and it only briefly previews the conclusion of the paper, if at all." \cite{mensh2017ten}.



\paragraph{ End of introduction:}

 
 Here you say what problem you are tackling: this should be made more clear: 
 \begin{enumerate}
    \item What is the missing gap and 
    \item Why is it exciting and important.
    \item What can be said/done / tested experimentally if we have modelled this, i.e. what consequences does it have or what hypothesis can be derived.  
 \end{enumerate}
Such a section is particularly important for a journal.  
