\documentclass[10pt,letterpaper]{article}
\usepackage[top=0.85in,left=2.75in,footskip=0.75in,marginparwidth=2in]{geometry}

% use Unicode characters - try changing the option if you run into troubles with special characters (e.g. umlauts)
\usepackage[utf8]{inputenc}

% clean citations
\usepackage{cite}


% hyperref makes references clicky. use \url{www.example.com} or \href{www.example.com}{description} to add a clicky url
\usepackage{nameref,hyperref}


\usepackage{todonotes}
\usepackage[capitalize]{cleveref}

% line numbers
\usepackage[right]{lineno}

% improves typesetting in LaTeX
\usepackage{microtype}
\DisableLigatures[f]{encoding = *, family = * }

% text layout - change as needed
\raggedright
\setlength{\parindent}{0.5cm}
\textwidth 5.25in 
\textheight 8.75in

% Remove % for double line spacing
%\usepackage{setspace} 
%\doublespacing

% use adjustwidth environment to exceed text width (see examples in text)
\usepackage{changepage}

% adjust caption style
\usepackage[aboveskip=1pt,labelfont=bf,labelsep=period,singlelinecheck=off]{caption}

% remove brackets from references
\makeatletter
\renewcommand{\@biblabel}[1]{\quad#1.}
\makeatother

% headrule, footrule and page numbers
\usepackage{lastpage,fancyhdr,graphicx}
\usepackage{epstopdf}
\pagestyle{myheadings}
\pagestyle{fancy}
\fancyhf{}
\rfoot{\thepage/\pageref{LastPage}}
\renewcommand{\footrule}{\hrule height 2pt \vspace{2mm}}
\fancyheadoffset[L]{2.25in}
\fancyfootoffset[L]{2.25in}

% use \textcolor{color}{text} for colored text (e.g. highlight to-do areas)
\usepackage{color}

% define custom colors (this one is for figure captions)
\definecolor{Gray}{gray}{.25}

% this is required to include graphics
\usepackage{graphicx}

% use if you want to put caption to the side of the figure - see example in text
\usepackage{sidecap}

% use for have text wrap around figures
\usepackage{wrapfig}
\usepackage[pscoord]{eso-pic}
\usepackage[fulladjust]{marginnote}
\reversemarginpar

% document begins here
\begin{document}
\vspace*{0.35in}

% title goes here:
\begin{flushleft}
{\Large
\textbf\newline{ Title of the paper is very important.
 }
}
\newline
% authors go here:
\\
Author 1\textsuperscript{1,*},
Author 1\textsuperscript{2},
Author 1\textsuperscript{1},
\\
\bigskip
\bf{1} Affiliation A
\\
\bf{2} Affiliation B
\\
\bigskip
* author1@uni.edu

\end{flushleft}

\section*{Abstract}


% now start line numbers
\linenumbers

% the * after section prevents numbering


\paragraph{In general:}

 \begin{itemize}
     \item highlight clearly what gap the paper is filling and how. 
     \item What makes it sexy. Grandfather should understand what the problem is and how we are going to tackle it.
     \item  Important for each experiment is why we did it. why do we actually perform these experiments and not others. 
 \end{itemize}

\section*{Introduction}


What is the question you are going to answer?
\begin{itemize}
    \item Why is it an exciting question (basically the problem statement)
    \item Why is this question important to our field
    \item How is my work going to help to answer it?
\end{itemize}

"I have a major question. How did I get to this question? Why is that important? That’s essentially the introduction. And then I have the second part which is this idea that now I’m going to lead you through how I answered it, and that’s our methods and results. So, I think the story part comes in putting your work into context of the field, of other people’s work, of why it’s important, and it’ll make your results much more compelling" \cite{mensh2017ten}. 


Readers might look at the title, they might skim your abstract and they might look at your figures, so we try to make our figures tell the story as much as possible.


\paragraph{How to structure a paragraph}
"For the whole paper, the introduction sets the context, the results present the content and the discussion brings home the conclusion" \cite{mensh2017ten}..

"In each paragraph, the first sentence defines the context, the body contains the new idea and the final sentence offers a conclusion" \cite{mensh2017ten}..




\paragraph{From 'Ten Simple Rules for structuring papers'}
"The introduction highlights the gap that exists in current knowledge or methods and why it is important. This is usually done by a set of progressively more specific paragraphs that culminate in a clear exposition of what is lacking in the literature, followed by a paragraph summarizing what the paper does to fill that gap.

As an example of the progression of gaps, a first paragraph may explain why understanding cell differentiation is an important topic and that the field has not yet solved what triggers it (a field gap). A second paragraph may explain what is unknown about the differentiation of a specific cell type, such as astrocytes (a subfield gap). A third may provide clues that a particular gene might drive astrocytic differentiation and then state that this hypothesis is untested (the gap within the subfield that you will fill). The gap statement sets the reader’s expectation for what the paper will deliver.

The structure of each introduction paragraph (except the last) serves the goal of developing the gap. Each paragraph first orients the reader to the topic (a context sentence or two) and then explains the “knowns” in the relevant literature (content) before landing on the critical “unknown” (conclusion) that makes the paper matter at the relevant scale. Along the path, there are often clues given about the mystery behind the gaps; these clues lead to the untested hypothesis or undeveloped method of the paper and give the reader hope that the mystery is solvable. The introduction should not contain a broad literature review beyond the motivation of the paper. This gap-focused structure makes it easy for experienced readers to evaluate the potential importance of a paper—they only need to assess the importance of the claimed gap.

The last paragraph of the introduction is special: it compactly summarizes the results, which fill the gap you just established. It differs from the abstract in the following ways: it does not need to present the context (which has just been given), it is somewhat more specific about the results, and it only briefly previews the conclusion of the paper, if at all." \cite{mensh2017ten}.



\paragraph{ End of introduction:}

 
 Here you say what problem you are tackling: this should be made more clear: 
 \begin{enumerate}
    \item What is the missing gap and 
    \item Why is it exciting and important.
    \item What can be said/done / tested experimentally if we have modelled this, i.e. what consequences does it have or what hypothesis can be derived.  
 \end{enumerate}
Such a section is particularly important for a journal.  



%%% This figure is floating around the text

% \marginpar{
% \vspace{.7cm} % adjust vertical position relative to text with \vspace{} - note that you can enter negative numbers to move margin caption up
% \color{Gray} % this gives caption a grey color to set it apart from text body
% \textbf{Figure \ref{fig1}. Example of a margin caption.} % note that \ref{fig1} refers to the corresponding wrapfigure
% Setting up your figure + caption like this looks fancy and does not disrupt the flow of the text. But it requires more manual adjustments (position, spacing, labeling) compared to using standard \LaTeX figure environments.
% }
% \begin{wrapfigure}[19]{l}{75mm}
% % the number in [] of wrapfigure is optional and gives the number of text lines that should be wrapped around the text. Adjust according to your figures height
% \includegraphics[width=75mm]{fig1.pdf}
% \captionsetup{labelformat=empty} % makes sure dummy caption is blank
% \caption{} % add dummy caption - otherwise \label won't work and figure numbering will not count up
% \label{fig1} % use \ref{fig1} to reference to this figure
% \end{wrapfigure} % avoid blank space here



% \subsection*{Page break in figures.}
% The standard floating figures in \LaTeX do not cope well with page breaks which can make it difficult to fit in large figures. One way to deal with this is to separate figure and caption but \verb!\caption{}! might still give troubles at page breaks. Figure 3 demonstrates a way to manually set up figure and caption such that it continues onto the next page.
% \vspace{.5cm} % set vertical space between text and figure
% \begin{adjustwidth}{-2in}{0in}
% \begin{flushright}
% \includegraphics[width=163mm]{fig3.pdf}
% \end{flushright}
% \justify 
% \color{Gray}
% \textbf {Figure 3. Example of a wide figure with multi-page caption.}
% \textbf{A}, Proin lectus ex, venenatis vel ornare eget, hendrerit tempus justo. Pellentesque molestie purus sed pretium tincidunt. Curabitur facilisis, orci vitae mollis fringilla, elit erat fermentum justo, nec luctus nunc sapien vel dolor. Cras enim justo, ullamcorper ut commodo at, posuere et ex. Fusce cursus sapien id augue maximus convallis. Praesent egestas massa in enim volutpat varius. In aliquam turpis urna, at elementum turpis eleifend at. \textbf{B}, Proin risus erat, tincidunt quis massa non, sollicitudin congue metus. Aliquam quis magna vulputate, posuere est eu, tempor nisi. Cras gravida tempus felis, vitae lacinia lacus volutpat quis. Pellentesque et eros eu mi suscipit tempus. Proin in augue scelerisque. \textbf{C}, Donec a tempor tortor, et dignissim enim. Cras in ipsum sed velit bibendum imperdiet. Aenean aliquet mauris maximus, sodales ligula sit amet, placerat felis. In tristique nisi eu risus rutrum, ac lacinia lorem cursus. Nunc eget condimentum purus. Maecenas imperdiet nisl eu accumsan gravida. \textbf{D}, Nullam tincidunt, magna sed auctor ultrices, leo mi eleifend velit, quis varius ex diam non tellus. Nam tincidunt vehicula turpis, ut euismod turpis elementum vel.
% \end{adjustwidth}


\section*{Materials and Methods}

\subsection*{Formulas.}
For mathematical formulas you should use the math environment. See this example:

\begin{center}
$f(A_{ik},A_{jk}) = min(A_{ik},A_{jk}) - C_{1} max(A_{ik},A_{jk}) e^{-C_{2}min(A_{ik},A_{jk})}$
\end{center}

\newpage

\section*{Results}

describe the logic and the reasoning behind these experiment and why one exp. follows logically on the other. 

E.g.: In order to .... we conducted exp 1 which results in .... This leads to Exp  2 which helps solving xxxx . ... 


an introduction  with problem description / hypothesis is something I expect for the all experiments . That woulds make it much easier to follow the logic of the paper.


\subsection*{Experiment 1}
\subsection*{Experiment 2}



% \subsection*{Standard floating figures.}
% Figure \ref{fig2} is wrapped into a standard floating environment. That means that \LaTeX will determine the exact placement of the figure. Even though you can state preferences (see code) it can be tricky to get the right placement - especially when working on very tight manuscripts. If you want exact placement, add \verb!\usepackage{float}! to this file's header and use [H] in the figure environment's placement options.


% \begin{figure}[ht] %s state preferences regarding figure placement here
% % use to correct figure counter if necessary
% %\renewcommand{\thefigure}{2}
% \includegraphics[width=\textwidth]{fig2.pdf}
% \caption{\color{Gray} \textbf{Example of a standard floating figure}. \textbf{A-F}, This figure is wrapped into the standard floating environment.}
% \label{fig2} % \label works only AFTER \caption within figure environment
% \end{figure}







\section*{Discussion}




%\clearpage

\section*{Supporting Information}
If you intend to keep supporting files separately you can do so and just provide figure captions here. Optionally make clicky links to the online file using \verb!\href{url}{description}!.

%These commands reset the figure counter and add "S" to the figure caption (e.g. "Figure S1"). This is in case you want to add actual figures and not just captions.
\setcounter{figure}{0}
\renewcommand{\thefigure}{S\arabic{figure}}

% You can use the \nameref{label} command to cite supporting items in the text.
\subsection*{S1 Figure}
\label{example_label}
{\bf Caption of Figure S1.} \textbf{A}, If you want to reference supporting figures in the text, use the \verb!\nameref{}!. command. This will reference the section's heading: \nameref{example_label}.

\subsection*{S2 Video}
\label{example_video}
{\bf Example Video.} Use \href{www.youtube.com}{clicky links} to the online sources of the files.

%\clearpage

\section*{Acknowledgments}
We thank just about everybody.

\nolinenumbers

%This is where your bibliography is generated. Make sure that your .bib file is actually called library.bib
\bibliography{library}

%This defines the bibliographies style. Search online for a list of available styles.
\bibliographystyle{abbrv}

\end{document}

