\section*{Results}

describe the logic and the reasoning behind these experiment and why one exp. follows logically on the other. 

E.g.: In order to .... we conducted exp 1 which results in .... This leads to Exp  2 which helps solving xxxx . ... 


an introduction  with problem description / hypothesis is something I expect for the all experiments . That woulds make it much easier to follow the logic of the paper.


\subsection*{Experiment 1}
\subsection*{Experiment 2}



% \subsection*{Standard floating figures.}
% Figure \ref{fig2} is wrapped into a standard floating environment. That means that \LaTeX will determine the exact placement of the figure. Even though you can state preferences (see code) it can be tricky to get the right placement - especially when working on very tight manuscripts. If you want exact placement, add \verb!\usepackage{float}! to this file's header and use [H] in the figure environment's placement options.


% \begin{figure}[ht] %s state preferences regarding figure placement here
% % use to correct figure counter if necessary
% %\renewcommand{\thefigure}{2}
% \includegraphics[width=\textwidth]{fig2.pdf}
% \caption{\color{Gray} \textbf{Example of a standard floating figure}. \textbf{A-F}, This figure is wrapped into the standard floating environment.}
% \label{fig2} % \label works only AFTER \caption within figure environment
% \end{figure}




